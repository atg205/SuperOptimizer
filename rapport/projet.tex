\documentclass[10pt,a4paper]{report}
\usepackage[utf8]{inputenc}
\usepackage[french]{babel}
\usepackage[T1]{fontenc}
\usepackage{amsmath}
\usepackage{amsfonts}
\usepackage{amssymb}
\usepackage{multirow}
\usepackage{graphicx}
\usepackage{caption}
\usepackage{hyperref}
\usepackage{svg}
\usepackage[left=2cm,right=2cm,top=2cm,bottom=2cm]{geometry}
\author{Karim ELLOUZE, Philipp HANUSSEK}
\title{Projet MOGPL : optimisation robuste dans l’incertain total}
\begin{document}
\maketitle
\newpage
\chapter{Linéarisation des critères maxmin et minmax regret}
\section{Programme linéare \textit{maxmin}}
\begin{align*}
\max \quad & t \\
\text{sous contraintes} \quad & t \leq \sum_{i=1}^{n} v_{i,1} x_i \\
& t \leq \sum_{i=1}^{n} v_{i,2} x_i \\
& \sum_{i=1}^{n} w_i x_i \leq W \\
& x_i \in \{0, 1\}, \quad \forall i = 1, \ldots, n
\end{align*}
Solution optimale trouvé avec Gurobi:
\begin{align*}
	x^* 	&= [2, 3, 4, 7, 8, 9]\\
	z(x^*) 	&= [66,66]
\end{align*}
\section{Programme linéaire \textit{minmax regret}}
\begin{align*}
\min \quad & t \\
\text{sous contraintes} \quad & t \geq z_1^* - \sum_{i=1}^{n} v_{i,1} x_i \\
& t \geq z_2^* - \sum_{i=1}^{n} v_{i,2} x_i \\
& \sum_{i=1}^{n} w_i x_i \leq W \\
& x_i \in \{0, 1\}, \quad \forall i = 1, \ldots, n
\end{align*}
$z_i^*$ étant la meilleure valeure pour le scénario i\\\\
Solution optimale trouvé avec Gurobi:
\begin{align*}
	x^{'*} 	&= [2, 3, 6, 7, 8, 9]\\
	z(x_1^*,x_2^*) &= [112,118]\\ 
	z(x^{'*}) 	&= [62,70]
\end{align*}
\section{Représentation dans le plan}
\section{Analyse de pérformance}
\begin{figure}[h!]
    \centering
    \includesvg[width=\linewidth]{figures/performance_ex1.svg}  
    \caption{Comparaison des temps d'exécution pour les modèles maxmin et minmax regret.}
    \label{fig:performance_ex1}
\end{figure}
Regret prend plus de temps que maxMin, parce qu'il calcule la valeur optimale $z^*$ pour chaque scénario d'abord. Le temps de calcul augmente pour les deux quand il y a plus des contraints (traduit par plusieurs scénarios). Les résultats trouvés qui montrent un temps supérieur de résolution pour $p=10$ à $p=20$ ne sont pas consistente avec ce qui est attendu et avec ce qui est observé dans l'exercice 2 et 3.
\chapter{Linéarisation du critère \textit{maxOWA}}
\section{$L_k(z)$ comme valeur optimale}
\textbf{Explication :} Étant donné que le vecteur $(z_{(1)}, \ldots, z_{(n)})$ est ordonné en ordre croissant, la somme $\sum_{i=1}^{k} z(i)$ correspond à la somme des $k$ plus petites valeurs de $z_i$ et donc à la valeur optimale du programme linéaire.
\section{Utiliser le dual pour trouver les composantes du $L(z)$ }
Primal:

\begin{align*}
\min & \sum_{i=1}^{n} a_{ik} z_i \\
\text{s.c.} \quad & \sum_{i=1}^{n} a_{ik} = k \\
& 0 \leq a_{ik} \leq 1, \quad i = 1, \dots, n
\end{align*}

Dual:

\begin{align*}
\max & \quad k \cdot r_k - \sum_{i=1}^{n} b_{ik} \\
\text{s.c.} \quad & r_k - b_{ik} \leq z_i, \quad \forall i = 1, \dots, n \\
& b_{ik} \geq 0, \quad \forall i = 1, \dots, n
\end{align*}
Trouver $L(z)$ avec $z=[2,9,6,8,5,4]$ (voir le code pour calculation par programmation linéaire):
\begin{align*}
L_1(z)&= 2\\
L_2(z)&= 6\\
L_3(z)&= 11\\
L_4(z)&= 17\\
L_5(z)&= 25\\
L_6(z)&= 34\\
\end{align*}
\section{Reécrire l'OWA}
Montrer que 
\begin{align*}
g(x)& =\sum_{i=1}^{n}w_iz_{(i)}(x)\\
&=\sum_{k=1}^{n}w'_k L_k(z(x))
\end{align*}
\textbf{Explication:}
\begin{align*}
\text{par définition: }z_{(k)}(x)&=L_k(z(x)) - L_{k-1}(z(x))\\
g(x)& =\sum_{i=1}^{n}w_iz_{(i)}(x)\\
&=w_1z_{(1)}+w_2z_{(2)}+\ldots+w_nz_{(n)}\\
&=w_1L_1(z(x))+w_2(L_2(z(x))-L_1(z(x)))+\ldots+w_n(L_n(z(x))-L_{n-1}(z(x))\\
&=(w_1-w_2)L_1(z(x))+(w_2-w_3)(L_2(z(x)))+\ldots+(w_{n-1}-w_{n})L_n\\
&=\sum^n_{k=1}w'_kL_k(z(x))
\end{align*}
\section{Formulation \textit{maxOWA} pour le problème SAD}
\begin{align*}
&\max && \sum_{k=1}^n w'_k \left( k \cdot r_k - \sum_{i=1}^n b_{ik} \right) \\
&\text{s.c.} && r_k - b_{ik} \leq z_i(x)\\
&&& b_{ik} \geq 0\\
&&& \sum_{i=1}^{n} w_i x_i \leq W, \\
&&& x_i \in \{0, 1\}, \quad i = 1, \dots, n.
\end{align*}
Solution optimale trouvé avec Gurobi:
\begin{align*}
	x^{*} 	&= [2, 3, 4, 7, 8, 9]\\
	z(x^{*}) 	&= [66, 66]
\end{align*}
\section{Formulation linéaire pour le critère de \textit{minOWA}}
Primal:\\\\
$L_k(z)=\sum_{i=1}^{k} r(x,s_{(i)})$, regret en ordre décroissant, donc:\\
\begin{align*}
    \max &\sum_{i=1}^n a_{ik} r(x,s_{(i)})\\
\text{s.c.} \quad & \sum_{i=1}^{n} a_{ik} = k \\
& 0 \leq a_{ik} \leq 1, \quad i = 1, \dots, n
\end{align*}
$r(x,s_{(i)}) = z_i^*-z_i(x)$\\\\
Dual:
\begin{align*}
\min & \quad k \cdot r_k + \sum_{i=1}^{n} b_{ik} \\
\text{s.c.} \quad & r_k + b_{ik} \geq r(x,s_{(i)}), \quad \forall i = 1, \dots, n \\
& b_{ik} \geq 0, \quad \forall i = 1, \dots, n
\end{align*}\\\\
Formulation entière:
\begin{align*}
\min &\sum_{k=1}^n w'_k \left( k \cdot r_k + \sum_{i=1}^n b_{ik} \right) \\
\text{s.c.} & \quad r_k + b_{ik} \geq r(x,s_{(i)}), \quad i = 1, \dots, n \\
& \quad b_{ik} \geq 0, \quad i = 1, \dots, n \\
& \sum_{i=1}^{n} w_i x_i \leq W\\
&x_i\in\{0,1\}\quad \forall i=1, \dots,n
\end{align*}
Solution optimale trouvé avec Gurobi:
\begin{align*}
	x^{'*} 	&= [2, 4, 5, 7, 8, 9]\\
	z(x_1^*,x_2^*) &= [112,118]\\ 
	z(x^{'*}) 	&= [62,70]
\end{align*}
\section{Analyse de performance}
\begin{figure}[h!]
    \centering
    \includesvg[width=\linewidth]{figures/performance_ex2.svg}  
    \caption{Comparaison des temps d'exécution pour les modèles maxOWA et minOWA des regrets.}
    \label{fig:performance_ex2}
\end{figure}
Comme dans l'exercice 1, on observe que minOWA regrets prend plus de temps que maxOWA. Le temps d'exection augmente avec le nombre de variables (projets) et le nombre de contraints (scénarios). En plus, on observe que le temps de résolution augment plus en ajoutant une contrainte qu'en ajoutant une variable. 
\chapter{Application à la recherche d’un chemin robuste dans un graphe}
\section{Modelisation du problème de plus court chemin comme problème du flot de coût minimum}

$G=(V,E)$, source: $s \in V$, puit: $t \in V$

$a(u,v)$: cout pour passer de u à v, $f(u,v)=1$ si $(u,v) \in $ {plus court chemin}\\\\
Programme linéaire:
\begin{align*}
\min&\sum_{(u,v)\in E} a(u,v)f(u,v)&\\
s.c.& \sum_{(u,v)\in E}f(u,v) = \sum_{(v,w)\in E}f(v,w) \quad \forall v \in V \setminus \{s, t\}& \text{, contrainte de conservation du flux}\\
    & \sum_{s,v \in E} f(s,v) >= 1 &\text{ , contrainte de flux sortant de la source}\\
    &f(u,v)\in (0,1) \quad \forall (u,v) \in E&
\end{align*}
\section{Application aux exemples}
Solution optimale trouvé avec Gurobi pour l'instance 1:
\begin{align*}
	x_1^{*} 	&= [a, b, d, f]\\
	x_2^*		&= [a, c, d, f]\\
	z(x^{*}) 	&= [8, 4]
\end{align*}
Pour l'instance 2:
\begin{align*}
	x_1^{*} 	&= [a, d, c, f, g]\\
	x_2^*		&= [a, c, e, g]\\
	z(x^{*}) 	&= [5, 6]
\end{align*}
\section{Programmes linéaires pour trouver un chemin robuste}
$x$ est une solution valide si:
\begin{align*}
           & \sum_{(u,v)\in E}f(u,v) = \sum_{(v,w)\in E}f(v,w) \quad \forall v \in V \setminus \{s, t\}& \text{, contrainte de conservation du flux}\\
            & \sum_{s,v \in E} f(s,v) >= 1 &\text{ , contrainte de flux sortant de la source}&\\
            &f(u,v)\in \{0,1\} \quad \forall (u,v) \in E&
\end{align*}
\subsection*{\textit{minmax}}
Comme on veut trouver le plus court chemin, on calcule le \textit{min} des solutions maximales pour un chemin $X$.

Programme linéaire pour $n$ scénarios:
\begin{align*}
\min\quad   & t&&\\
s.c.\quad   &t\ge\sum_{(u,v)\in E} a_i(u,v)f(u,v)&i\in\{1,\ldots,n\}\\
                & x \in X\\
            &f(u,v)\in \{0,1\} \quad \forall (u,v) \in E&
\end{align*}
\subsection*{\textit{minmax} des regrets}
Programme linéaire pour $n$ scénarios:
\begin{align*}
\min\quad   & t&&\\
s.c.\quad   &t\ge\sum_{(u,v)\in E} \bigl(a_i(u,v)f(u,v)\bigl)-z_i^*&i\in\{1,\ldots,n\}\\
                & x \in X\\
            &f(u,v)\in \{0,1\} \quad \forall (u,v) \in E&
\end{align*}
\subsection*{\textit{minOWA}}
\begin{align*}
\min            &\sum_{k=1}^n w'_k \left( k \cdot r_k + \sum_{i=1}^n b_{ik} \right) \\
\text{s.c.}     & \quad r_k + b_{ik} \geq z_i(x) \quad i = 1,\ldots,n\\
                & x \in X\\
                & \quad b_{ik} \geq 0, \quad i = 1, \dots, n \\
\end{align*}
\subsection*{\textit{minOWA} des regrets}
\begin{align*}
\min &\sum_{k=1}^n w'_k \left( k \cdot r_k + \sum_{i=1}^n b_{ik} \right) \\
\text{s.c.} & \quad r_k + b_{ik} \geq r(x,s_{(i)}), \quad i = 1, \dots, n \\
& \quad b_{ik} \geq 0, \quad i = 1, \dots, n \\
& \quad x \in X
\end{align*}
\subsection*{Résultats}
\subsubsection{minMax et minMax des regrets}
\begin{table}[h!]
\centering
\renewcommand{\arraystretch}{1.5} % Augmente l'espacement vertical dans le tableau
\begin{tabular}{|c|c|c|c|c|}
\hline
\textbf{Instance} & \textbf{Méthode}        & \textbf{Chemin}  & \textbf{Valeur de la fonction objectif} & \textbf{Vecteur\_z} \\ \hline
\multirow{2}{*}{Instance 0} 
                  & minmax                  & a,b,d,f          & 9             & [8, 9]              \\ \cline{2-5}
                  & minMax des regrets      & a,b,e,f          & 3             & [11, 7]             \\ \hline
\multirow{2}{*}{Instance 1} 
                  & minmax                  & a,b,e,g          & 10            & [10, 10]            \\ \cline{2-5}
                  & minMax des regrets      & a,b,e,g          & 5             & [10, 10]            \\ \hline
\end{tabular}
\caption{Comparaison des résultats des méthodes minmax et minMax des regrets.}
\label{tab:resultats}
\end{table}
\subsubsection{minOWA et minOWA des regrets}
\begin{table}[h!]
\centering
\renewcommand{\arraystretch}{1.5}
\begin{tabular}{|c|c|c|c|c|c|}
\hline
\textbf{Méthode}        & \textbf{Instance} & \textbf{\(k\)} & \textbf{Chemin} & \textbf{\( \text{Vecteur\_z} \)} & \textbf{Valeur de la fonction objectif} \\ \hline
\multirow{8}{*}{minOWA} & \multirow{4}{*}{0} 
                        & 2  & a,b,d,f          & [8, 9]        & 26 \\ \cline{3-6} 
                        &                       & 4  & a,b,d,f          & [8, 9]        & 44 \\ \cline{3-6} 
                        &                       & 8  & a,b,d,f          & [8, 9]        & 80 \\ \cline{3-6} 
                        &                       & 16 & a,b,d,f          & [8, 9]        & 152 \\ \cline{2-6} 
                        & \multirow{4}{*}{1} 
                        & 2  & a,b,e,g          & [10, 10]      & 30 \\ \cline{3-6} 
                        &                       & 4  & a,b,e,g          & [10, 10]      & 50 \\ \cline{3-6} 
                        &                       & 8  & a,b,e,g          & [10, 10]      & 90 \\ \cline{3-6} 
                        &                       & 16 & a,b,e,g          & [10, 10]      & 170 \\ \hline
\multirow{8}{*}{minOWA des regrets} 
                        & \multirow{4}{*}{0} 
                        & 2  & a,b,e,f          & [11, 7]       & 9 \\ \cline{3-6} 
                        &                       & 4  & a,b,e,f          & [11, 7]       & 15 \\ \cline{3-6} 
                        &                       & 8  & a,b,e,f          & [11, 7]       & 27 \\ \cline{3-6} 
                        &                       & 16 & a,b,e,f          & [11, 7]       & 51 \\ \cline{2-6} 
                        & \multirow{4}{*}{1} 
                        & 2  & a,d,f,g          & [6, 12]       & 13 \\ \cline{3-6} 
                        &                       & 4  & a,b,e,g          & [10, 10]      & 24 \\ \cline{3-6} 
                        &                       & 8  & a,b,e,g          & [10, 10]      & 44 \\ \cline{3-6} 
                        &                       & 16 & a,b,e,g          & [10, 10]      & 84 \\ \hline
\end{tabular}
\caption{Comparaison des méthodes minOWA et minOWA des regrets.}
\label{tab:methods}
\end{table}
\begin{table}[h!]
\centering
\begin{tabular}{|c|c|c|}
\hline
Instance & \( z^* \) \\
\hline
0 & [8, 4] \\
1 & [5, 6] \\
\hline
\end{tabular}
\caption{Valeurs de \( z^* \) pour chaque instance}
\label{tab:z_star}
\end{table}
On observe dans le \autoref{tab:methods} qu'il y a un seul changement de chemin dans minOWA des regrets pour l'instance 1. Ce changement est due au fait que lorsque $k=2$ les regrets sont de $[1,6]$, donc pour une faible valeur $k$ le chemin \textit{a,d,f,g} est optimale, mais si on augmente $k$, le vecteur des regrets $z=[5,4]$ correspondant au chemin \textit{a,b,e,g} devient l'optimal. Donc pour des valeurs plus élevés de $k$ on cherche un vecteur (de regrets) avec des valeurs qui sont assez proches.

%supposons que $z_1-z_1^*>z_2-z_2^*$. Initialement, la valeur de la fonction objectif est $k(z_1-z_1^*)+z_2-z_2^*$, donc c'est le minimum. Soit $z'=[z'_1, z'_2]$ une solution qui initialemment n'est pas optimale. On a donc $k(z_1-z_1^*)+z_2-z_2^* <k(z'_1-z'_1^*)+z'_2-z'_2^*$. Si $z_1-z_1^*>z'_1-z'_1^*$
\section{Analyse de pérformance}
\begin{figure}[h!]
    \centering
    \includesvg[width=\linewidth]{figures/ex3/performance_comparison_all}  
    \caption{Comparaison des temps d'exécution pour les modèles minMax, minMaxRegret, minOWA et minOWA des regrets.}
    \label{fig:performance_ex3}
\end{figure}
On observe que les méthodes calculant le regret (minMaxRegret, minOWAregret) ont un temps moyen d'exécution plus important à cause du calcul de $z^*$. On observe que minOWA prend plus de temps que minMax et que minOWARegret prend plus de temps que minMaxRegret parce qu'on utilise plus de variables. Dans l'OWA il y a $n$ variables $r_k$ et $n^2$ variables $b_{ik}$ et pour maxMin il y a $p$ variables. Pour les valeurs $n$ et $p$ prises, $n^2+n$ est toujours supérieur à $p$.
\end{document}
